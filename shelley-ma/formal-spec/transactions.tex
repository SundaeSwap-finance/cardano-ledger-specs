\begin{figure*}[t!]
  %
  \emph{Abstract types}
  %
  \begin{equation*}
    \begin{array}{r@{~\in~}l@{\qquad=\qquad}lr}
      \var{s_{v1}} & \ScriptNI & \ScriptMSig \\
      \var{s_{v2}} & \Timelock & \text{extended script language}
    \end{array}
  \end{equation*}
  %
  \emph{Derived types}
  %
  \begin{equation*}
    \begin{array}{r@{~\in~}l@{\qquad=\qquad}lr}
      \var{txout} & \TxOut & \Addr \times \hldiff{\ValClass} \\
%      & \text{tx outputs}
      \var{s} & \Script & \hldiff{\ScriptNI \uniondistinct \Timelock}
%      & \text{scripts}
      \\
    \end{array}
  \end{equation*}
  %
  \emph{Transaction Type}
  %
  \begin{equation*}
    \begin{array}{r@{~~}l@{~~}l@{\qquad}l}
      \var{txbody} ~\in~ \TxBody ~=~
      & \powerset{\TxIn} & \fun{txinputs}& \text{inputs}\\
      &\times ~(\Ix \mapsto \TxOut) & \fun{txouts}& \text{outputs}\\
      & \times~ \seqof{\DCert} & \fun{txcerts}& \text{certificates}\\
       & \times ~\hldiff{\ValClass}  & \hldiff{\fun{mint}} &\text{value minted}\\
       & \times ~\Coin & \fun{txfee} &\text{non-script fee}\\
       & \times ~\hldiff{\Slot^? \times \Slot^?} & \hldiff{\fun{txvldt}} & \text{validity interval}\\
       & \times~ \Wdrl  & \fun{txwdrls} &\text{reward withdrawals}\\
       & \times ~\Update^?  & \fun{txUpdates} & \text{update proposals}\\
       & \times ~\AuxiliaryDataHash^? & \fun{txADhash} & \text{auxiliary data hash}
    \end{array}
  \end{equation*}
  %
  \begin{equation*}
    \begin{array}{r@{~~}l@{~~}l@{\qquad}l}
      \var{ad} ~\in~ \AuxiliaryData ~=~
      & \powerset{\Script} & \fun{scripts}& \text{Optional scripts}\\
      & \times ~\Metadata^? & \fun{md} & \text{metadata}
    \end{array}
  \end{equation*}
  %
  \begin{equation*}
    \begin{array}{r@{~~}l@{~~}l@{\qquad}l}
      \var{tx} ~\in~ \Tx ~=~
      & \TxBody & \fun{txinputs}& \text{Transaction body}\\
      &\times ~\TxWitness & \fun{txouts}& \text{Witnesses}\\
      & \times ~\AuxiliaryData^? & \fun{txADhash} & \text{Auxiliary data}
    \end{array}
  \end{equation*}
  %
  \emph{Accessor Functions}
  \begin{equation*}
    \begin{array}{r@{~\in~}lr}
      \fun{getValue} & \TxOut \to \ValClass & \text{output value} \\
      \fun{getAddr} & \TxOut \to \Addr & \text{output address}
    \end{array}
  \end{equation*}
  \caption{Type Definitions used in the UTxO transition system}
  \label{fig:defs:utxo-shelley}
\end{figure*}

\section{Transactions}
\label{sec:transactions}

This section describes the changes that are made to the
transaction structure to support both timelock script and native multi-asset (MA)
functionality in the Cardano ledger.

\subsection*{New Output Type}

A key change needed to introduce MA functionality in the Mary era is changing the type of
the transaction and UTxO outputs to contain multi-asset values to accommodate
transacting with these types of assets natively, using the same ledger accounting
scheme as is used for Ada. We set up the infrastructure for this
change in transition system update common to both Allegra and Mary, ie. ShelleyMA. That is,
$\TxOut$ now contains a term that is a member of the $\ValClass$ type (which
could be a $\Coin$, $\Value$, or another type, see Section \ref{sec:coin-ma}),
rather than only $\Coin$.

In the Allegra era, $\ValClass$ is instantiated with $\Coin$, so there is no
multi-asset support in the UTxO or the transactions. In the Mary era, it
is changed to $\Value$ in order to add multi-asset support.

\subsection*{New Script Type}

The multi-signature scripting language has been renamed to $\ScriptNI$ and
the type $\Script$ has been extended to include a new scripting language,
$\Timelock$. The two types of scripts which make up the $\Script$ type are both
native, meaning
that they are evaluated by the ledger directly. We specify the evaluation
function for the additional script type in section~\ref{sec:timelock-lang}.

Both types of scripts can be used for the same purposes, which means, for this
spec, as

\begin{itemize}
  \item output-locking scripts,
  \item certificate validation,
  \item reward withdrawals, or
  \item as minting scripts (see below).
\end{itemize}

\subsection*{The Mint Field}

The body of a transaction in the ShelleyMA era contains one additional
field, the $\fun{mint}$ field.
The $\fun{mint}$ field is a term of type $\ValClass$, which contains
tokens the transaction is putting into or taking out of
circulation. Here, by "circulation", we mean specifically "the UTxO on the
ledger". Since the administrative fields cannot contain tokens other than Ada,
and Ada cannot be minted (this is enforced by the UTxO rule, see Figure~\ref{fig:rules:utxo-shelley}),
they are not affected in any way by minting.

Putting tokens into circulation is done with positive values in the $\Quantity$
fields $\fun{mint}$ field, and taking tokens out of circulation can be done
with negative quantities.

A transaction cannot simply mint arbitrary tokens. In the Mary era, restrictions on
multi-asset tokens are imposed, for each asset with ID $\var{pid}$, by a script
with the hash $\var{pid}$. Whether a minting transaction adheres to the restrictions
prescribed by preimage script is verified as part of the processing of the transaction.
The minting mechanism is detailed in Section~\ref{sec:utxo}.

Note that the $\fun{mint}$ field exists in both Allegra and Mary eras,
but can only be used in the Mary era, when multi-assets are introduced.
In Figure \ref{fig:minted} we give two versions of a function $\fun{minted}$,
which is used to inspect the $\fun{mint}$ field and return the set of hashes
of scripts that must be run to validate the minting of the tokens therein :

\begin{itemize}
  \item For the Allegra version of the function, which operates on $\Coin$ as the
  type with which $\ValClass$ is instantiated, no scripts need to be run
  to validate minting, since no minting of Ada is ever allowed.

  \item For the Mary version, the set of $\fun{PolicyID}$s of the
  assets in the $\Value$ term of the $\fun{mint}$ field is returned.
\end{itemize}

\begin{figure*}[t!]
  \emph{ShelleyMA Abstract $\fun{minted}$ Field Type}
  %
  \begin{align*}
    \fun{minted}~\in& \TxBody~\to~\powerset{ScriptHash}
  \end{align*}
  %
  \emph{Allegra Instance}
  %
  \begin{align*}
    \fun{minted}~\wcard~=& \emptyset
  \end{align*}
  %
  \emph{Mary Instance}
  %
  \begin{align*}
    \fun{minted}~\var{txb}~=& \supp~(\fun{mint}~{txb})
  \end{align*}
  \caption{$\fun{minted}$ Field in ShelleyMA, Allegra, and Mary}
  \label{fig:minted}
\end{figure*}

\subsection*{Transaction Body}

The following changes were made to $\TxBody$:

\begin{itemize}
  \item a change in the type of $\TxOut$ --- instead of
$\Coin$, the transaction outputs now have type $\ValClass$.
  \item the addition of the $\fun{mint}$ field to the transaction body
  \item the time-to-live slot number (which had the accessor $\fun{txttl}$),
  has been replaced with a validity interval with accessor $\fun{txvldt}$,
  both endpoints of which are optional
\end{itemize}

The only change to the types related to transaction witnessing is the addition
of minting policy scripts to the underlying $\Script$ type, so we do not include the
whole $\Tx$ type here.

\subsection*{Auxiliary Data}

The auxiliary data section of the transaction consists of two parts:
$\fun{md}$ which is the same metadata type used in
Shelley, and $\fun{scripts}$, which is used to allow transactions to
contain optional scripts that are not used for witnessing. The
intended purpose of this field is to give users the ability to
transmit a script that belongs to a script address via this
auxiliary data. There are no restrictions on this field however, and users
can use it to put arbitrary scripts into the auxiliary data.
