\section{Blockchain layer}
\label{sec:chain}

\newcommand{\Proof}{\type{Proof}}
\newcommand{\Seedl}{\mathsf{Seed}_\ell}
\newcommand{\Seede}{\mathsf{Seed}_\eta}
\newcommand{\activeSlotCoeff}[1]{\fun{activeSlotCoeff}~ \var{#1}}
\newcommand{\slotToSeed}[1]{\fun{slotToSeed}~ \var{#1}}

\newcommand{\T}{\type{T}}
\newcommand{\vrf}[3]{\fun{vrf}_{#1} ~ #2 ~ #3}
\newcommand{\verifyVrf}[4]{\fun{verifyVrf}_{#1} ~ #2 ~ #3 ~#4}

\newcommand{\HashHeader}{\type{HashHeader}}
\newcommand{\HashBBody}{\type{HashBBody}}
\newcommand{\bhHash}[1]{\fun{bhHash}~ \var{#1}}
\newcommand{\bHeaderSize}[1]{\fun{bHeaderSize}~ \var{#1}}
\newcommand{\bSize}[1]{\fun{bSize}~ \var{#1}}
\newcommand{\bBodySize}[1]{\fun{bBodySize}~ \var{#1}}
\newcommand{\OCert}{\type{OCert}}
\newcommand{\BHeader}{\type{BHeader}}
\newcommand{\BHBody}{\type{BHBody}}

\newcommand{\bheader}[1]{\fun{bheader}~\var{#1}}
\newcommand{\hsig}[1]{\fun{hsig}~\var{#1}}
\newcommand{\bprev}[1]{\fun{bprev}~\var{#1}}
\newcommand{\bhash}[1]{\fun{bhash}~\var{#1}}
\newcommand{\bvkcold}[1]{\fun{bvkcold}~\var{#1}}
\newcommand{\bseedl}[1]{\fun{bseed}_{\ell}~\var{#1}}
\newcommand{\bprfn}[1]{\fun{bprf}_{n}~\var{#1}}
\newcommand{\bseedn}[1]{\fun{bseed}_{n}~\var{#1}}
\newcommand{\bprfl}[1]{\fun{bprf}_{\ell}~\var{#1}}
\newcommand{\bocert}[1]{\fun{bocert}~\var{#1}}
\newcommand{\bnonce}[1]{\fun{bnonce}~\var{#1}}
\newcommand{\bleader}[1]{\fun{bleader}~\var{#1}}
\newcommand{\hBbsize}[1]{\fun{hBbsize}~\var{#1}}
\newcommand{\bbodyhash}[1]{\fun{bbodyhash}~\var{#1}}
\newcommand{\overlaySchedule}[4]{\fun{overlaySchedule}~\var{#1}~\var{#2}~{#3}~\var{#4}}

\newcommand{\PrtclState}{\type{PrtclState}}
\newcommand{\PrtclEnv}{\type{PrtclEnv}}
\newcommand{\OverlayEnv}{\type{OverlayEnv}}
\newcommand{\VRFState}{\type{VRFState}}
\newcommand{\NewEpochEnv}{\type{NewEpochEnv}}
\newcommand{\NewEpochState}{\type{NewEpochState}}
\newcommand{\PoolDistr}{\type{PoolDistr}}
\newcommand{\BBodyEnv}{\type{BBodyEnv}}
\newcommand{\BBodyState}{\type{BBodyState}}
\newcommand{\RUpdEnv}{\type{RUpdEnv}}
\newcommand{\ChainEnv}{\type{ChainEnv}}
\newcommand{\ChainState}{\type{ChainState}}
\newcommand{\ChainSig}{\type{ChainSig}}

No changes to translate data between old and new types within the ledger or chain
state are required to transition from the
Shelley era to the Allegra era. Allegra changes are to the rules for
validating a new type of script, so we do not give a translation function here.
Note that while the multi-signature scripts are deprecated in the Allegra era,
the timelock scripts are intended to be encoding-compatible with the MSig ones.
That means that
For this reason,

In Figure~\ref{fig:functions:to-ma}, we give the function $\fun{translateEra}$ that
specifies the translation of an Allegra-era chain state into a chain state that provides multi-asset support
(the Mary era).
We use $\ChainState_{\mathsf{Allegra}}$ to denote the type of the chain state
in the Allegra era, and $\ChainState$ for the Mary era chain state.
We use the notation $\var{chainstate}_x$ to represent
variable $x$ in the chain state. We do not specify the variables that remain
unchanged during the transition.
The only part of the state that is affected by the transition is the UTxO, as
each UTxO map entry must be updated to contain a $\Value$ instead of $\Coin$.

%%
%% Figure - Allegra to Mary Transition
%%
\begin{figure}[htb]
  \begin{align*}
      & \fun{translateEra} ~\in~ \ChainState_{\mathsf{Allegra}}  \to \ChainState  \\
      & \fun{translateEra}_{\var{utxo}} ~=~ \{~ \var{txin} \mapsto (a,\fun{inject}~c) ~\vert~
      \var{txin} \mapsto \var{(a,c)}\in ~\var{utxo}~\}
  \end{align*}
  \caption{Allegra to Multi-Asset Chain State Transtition}
  \label{fig:functions:to-ma}
\end{figure}
