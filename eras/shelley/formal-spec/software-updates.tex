\section{Software Updates}
\label{sec:software-updates}

Updates to the software will include increasing the protocol version.
An increase in the major version indicates a hard fork, and the minor version a soft fork
(meaning old software can validate but not produce new blocks).

The current protocol version ($\ProtVer$) is a member of the protocol parameters.
It represents a specific version of the \textit{ledger rules}.
If $\var{pv}$ changes, this document may have to be updated with
the new rules and types if there is a change in the logic.
If there is a change in the transition rules, nodes must have
software installed that can implement these rules at the epoch boundary
when the protocol parameter adoption occurs.
Switching to using these new rules is mandatory in the sense that
if the nodes do not have the applications implementing them, this
will prevent a user from reading and writing to the ledger.

Applications must sometimes support \textit{several different versions}
of ledger rules in order to accommodate the timely switch of the $\ProtVer$ at the epoch boundary.
In this situation, the newest protocol version a node is ready to use is included in the block
header of the blocks it produces, see \ref{sec:defs-blocks}. This is either:

\begin{itemize}
\item the current version (if there is no protocol version update pending or the node
has not updated to an upcoming software version capable of of implementing a
newer protocol version), or
\item the next protocol version,
(if the node has updated its software, but the current protocol version on the
ledger has not been updated yet).
\end{itemize}

Stake pools have some agency in the process of adoption of
new protocol versions. They may refuse to download and install updates.
Since software updates cannot be \textit{forced} on the users, if the majority of
users do not perform an update which allows the switch to the next $\ProtVer$,
it cannot happen.

Note that if there is a \textit{new protocol version} implemented by new
software, the core nodes can monitor how many nodes are ready to use the new
protocol version via the block headers.
Once enough nodes are ready for the new protocol version, this
may now be updated as well (by the mechanism in described in 
Section \ref{sec:update}).
