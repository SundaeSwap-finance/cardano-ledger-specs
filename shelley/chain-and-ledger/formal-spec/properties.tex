\newcommand{\Val}{\fun{Val}}
\newcommand{\POV}[1]{\ensuremath{\mathsf{PresOfVal}(\mathsf{#1})}}
\newcommand{\DBE}[2]{\ensuremath{\mathsf{DBE}({#1},~{#2})}}
\newcommand{\DGO}[2]{\ensuremath{\mathsf{DGO}({#1},~{#2})}}
\newcommand{\transtar}[2]{\xlongrightarrow[\textsc{#1}]{#2}\negthickspace^{*}}

\section{Properties}
\label{sec:properties}

This section describes the properties that the ledger should have. The goal is to
to include these properties in the executable specification to enable e.g.
property-based testing or formal verification.

% TODO - the hand proofs of preservation of ada and non-negative deposit
% pot are woefully out of date, mostly due to the removal of decaying deposits.
%
\subsection{Preservation of Value}
\label{sec:preservation-of-value}

As visualized in Figure~\ref{fig:fund-preservation},
the total amount of lovelace in any given chain state
$\var{s}\in\ChainState$ is completely contained within the values of the six
variables:

\begin{tabular}{||l|l|l|l||}\hline\hline

  \textbf{Variable} & \textbf{Name in Figure~\ref{fig:fund-preservation}}
                    & \textbf{Nesting Inside Chain State} & \textbf{Kind} \\ \hline
  utxo & circulation & s.nes.es.ls.utxoSt & Map over Lovelace Values  \\ \hline
  deposits & deposits &  s.nes.es.ls.utxoSt & Lovelace Value ($\Coin$) \\ \hline
  fees & fees &  s.nes.es.ls.utxoSt & Lovelace Value ($\Coin$) \\ \hline
  rewards & reward accounts & s.nes.es.ls.dpstate.dstate  & Lovelace Value ($\Coin$)  \\ \hline
  treasury & treasury &  s.nes.es.acnt  & Lovelace Value ($\Coin$) \\ \hline
  reserves & reserves & s.nes.es.acnt & Map over Lovelace Values \\ \hline
  \hline
\end{tabular}

\noindent
Notice that $\var{deposits}$, $\var{fees}$, $\var{treasury}$, and $\var{reserves}$
are all single lovelace values, while $\var{utxo}$, and $\var{rewards}$ are
maps whose values are lovelace.

We define the \emph{Lovelace Value} of a given chain state as:
\begin{definition}[Lovelace Value]
  \label{def:val}
  \begin{equation*}
    \Val(s~\in~\var{State}) =
        \Val(\var{utxo}) +
            \Val(\var{deposits}) +
            \Val(\var{fees}) +
            \Val(\var{reserves}) +
            \Val(\var{treasury}) +
            \Val(\var{rewards})
  \end{equation*}
  where
  \begin{equation*}
      \Val(x \in \Coin) = x
  \end{equation*}
  \begin{equation*}
      \Val((\wcard\mapsto (y \in \Coin))^{*}) = \sum y
  \end{equation*}
\end{definition}

\noindent
For any state that is used in a given subtransition of $\mathsf{CHAIN}$,
we define $\Val{}$ in an analogous way, setting the value of any variable that is not explicitly
represented in the state to zero.
For example, given $\var{utxoSt}\in\UTxOState$,
\begin{equation*}
  \Val(\var{utxoSt}) =
  \left(\sum_{\wcard\mapsto(\wcard,~v)\in\var{utxo}}v\right) + \var{deposits} + \var{fees}
\end{equation*}

\noindent
The key property that we want to prove is that no semantic transition changes the value that
is captured in the state ($\Val{s}$).
This property is easy to state: intuitively,
the \emph{Lovelace Value}before the transition is the same as the
\emph{Lovelace Value} after that transition.

\begin{theorem}[Preservation of Value]
  \label{thm:chain-pres-of-value}
  For all environments $e$, blocks $b$, and states $s$, $s'$, if
  \begin{equation*}
    e\vdash s\trans{\hyperref[fig:rules:chain]{chain}}{b}s'
  \end{equation*}
  then
  \begin{equation*}
    \Val(s) = \Val(s')
  \end{equation*}
\end{theorem}

\noindent
We will prove the soundness of Theorem~\ref{thm:chain-pres-of-value} via a few lemmas.

\begin{lemma}
  \label{lemma:value-sum-pres-1}
  For any mapping $m:A\mapsto\Coin$ and set $s\in\powerset{A}$,
  \begin{equation*}
    \Val(\var{m}) = \Val(s\subtractdom m) + \Val(s\restrictdom m)
  \end{equation*}
\end{lemma}
\begin{proof}
  easy
\end{proof}

\begin{lemma}
  \label{lemma:value-sum-pres-2}
  For any mappings $m_1, m_2:A\mapsto\Coin$,
  if $\dom{m_1}\cap\dom{m_2}=\emptyset$,
  then
  \begin{equation*}
    \Val(m_1\cup m_2) = \Val(m_1) + \Val(m_2)
  \end{equation*}
\end{lemma}
\begin{proof}
  easy
\end{proof}

\begin{lemma}
  \label{lemma:utxo-pres-of-value}
  For all environments $e$, transactions $t$, and states $s$, $s'$, if
  \begin{equation*}
    e\vdash s\trans{\hyperref[fig:rules:utxo-shelley]{utxo}}{t}s'
  \end{equation*}
  then
  \begin{equation*}
    \Val(s) + w = \Val(s')
  \end{equation*}
  where $w = \fun{wbalance}~(\fun{txwdrls}~{t})$.
\end{lemma}

\begin{proof}
  The proof is essentially unfolding the definition of the predicate
  \begin{equation}
    \label{cons-is-prod}
    \consumed{pp}{utxo}{t} = \produced{pp}{stpools}{t}
  \end{equation}
  and applying a little algebra.
%
If we let:
  \begin{equation*}
    \begin{array}{r@{~=~}l}
      k & \keyRefunds{pp}{stkCreds}{t} \\
      f & \txfee{t} \\
      d & \totalDeposits{pp}{stpools}{(\txcerts{t})} \\
    \end{array}
  \end{equation*}
  then equation~\ref{cons-is-prod} can be rewritten as:
  \begin{equation*}
    \Val(\txins{t} \restrictdom{\var{utxo}}) + w + k = \Val(\outs{t}) + f + d
  \end{equation*}
  where $\outs{}$ is defined in Figure~\ref{fig:functions:utxo} and returns a value of type $\UTxO$.
  Therefore, moving $k$ to the right and adding $\txins{t} \subtractdom{\var{utxo}}$ to each side,
  \begin{equation*}
    \Val(\txins{t} \restrictdom{\var{utxo}}) + \Val(\txins{t} \subtractdom{\var{utxo}}) + w
    = \Val(\outs{t}) + f + d - k + \Val(\txins{t} \subtractdom{\var{utxo}})
  \end{equation*}
  (Though not needed for the proof at hand,
  note that $d-k$ is non-negative since the deposits will always be large enough to cover
  the current obligation. See Theorem~\ref{thm:non-neg-deposits}.)
%
  It then follows that:
  \begin{equation*}
    \begin{array}{r@{~=~}lr}
      \Val(\var{utxo}) + w
    & \Val(\outs{t}) + f + d - k + \Val(\txins{t} \subtractdom{\var{utxo}})
    & \text{(by Lemma~\ref{lemma:value-sum-pres-1})}
    \\
    & \Val((\txins{t} \subtractdom{\var{utxo}})\cup\outs{t}) + (d - k) + f
    & \text{(by Lemma~\ref{lemma:value-sum-pres-2})}
    \end{array}
  \end{equation*}
  Note that in order to apply Lemma~\ref{lemma:value-sum-pres-2} above,
  it must be true that $(\txins{t} \subtractdom{\var{utxo}})$ and $(\outs{t})$
  have disjoint domains, which follows from the uniqueness of the transaction IDs.

  Therefore, by adding the deposits and fees from $s$ to the equality above,
  it follows that $\Val(s) + w = \Val(s')$.
\end{proof}

\begin{lemma}
  \label{lemma:deleg-pres-of-value}
  For all environments $e$, transactions $c$, and states $s$, $s'$, if
  \begin{equation*}
    e\vdash s\trans{\hyperref[fig:delegation-rules]{deleg}}{c}s'
  \end{equation*}
  then
  \begin{equation*}
    \Val(s) = \Val(s')
  \end{equation*}
\end{lemma}

\begin{proof}
  The only variable with value in this transition is \var{rewards}.
  Only two of the rules in $\mathsf{DELEG}$ can change \var{rewards},
  namely $\mathsf{Deleg{-}Reg}$ and $\mathsf{Deleg{-}Dereg}$.
  However, $\mathsf{Deleg{-}Reg}$ only adds a zero value,
  and $\mathsf{Deleg{-}Dereg}$ only removes a zero value.
\end{proof}

\begin{lemma}
  \label{lemma:delegs-pres-of-value}
  For all environments $e$, certificates $\Gamma$, and states $s$, $s'$, if
  \begin{equation*}
    e\vdash s\trans{\hyperref[fig:rules:delegation-sequence]{delegs}}{\Gamma}s'
  \end{equation*}
  then
  \begin{equation*}
    \Val(s) = \Val(s') + w
  \end{equation*}
  where $w = \fun{wbalance}~(\fun{txwdrls}~{t})$,
  and $t$ is the transaction in the environment $e$.
\end{lemma}

\begin{proof}
  The proof is by induction on the length of $\Gamma$.
  Note that the only variable with value in this transition is \var{rewards}.

  \vspace{2ex}
  \noindent
  \emph{In the base case}, we look at the rule $\mathsf{Seq{-}delg{-}base}$.
  Since $\var{wdrls}\subseteq\var{rewards}$, then
  $\var{rewards} = \var{wdrls}\cup\var{(\var{rewards}\setminus\var{wdrls})}$.
%
  Therefore
  \begin{equation*}
    \begin{array}{r@{~=~}lr}
      \Val{(\var{rewards})}
      & \Val{(\var{rewards}\setminus\var{wdrls})} + \Val{(\var{wdrls})}
      & \text{by Lemma~\ref{lemma:value-sum-pres-2}}
      \\
      & \Val{(\var{rewards}\setminus\var{wdrls})} + w
      & \text{by definition}
      \\
      & \Val\left(\var{rewards}\unionoverrideRight\{(w, 0) \mid w \in \dom \var{wdrls}\}\right) + w
    \end{array}
  \end{equation*}
  Therefore $\Val(s) = \Val(s')$.

  \vspace{2ex}
  \noindent
  \emph{In the inductive case}, we look at the rule $\mathsf{Seq{-}delg{-}ind}$.
  In this case, the lemma then follows directly from Lemma~\ref{lemma:deleg-pres-of-value}.
\end{proof}

\begin{lemma}
  \label{lemma:poolreap-pres-of-value}
  For all environments $e$, epoch $\epsilon$, and states $s$, $s'$, if
  \begin{equation*}
    e\vdash s\trans{\hyperref[fig:rules:pool-reap]{poolreap}}{\epsilon}s'
  \end{equation*}
  then
  \begin{equation*}
    \Val(s) = \Val(s')
  \end{equation*}
\end{lemma}

\begin{proof}
  The $\mathsf{POOLREAP}$ value is contained in
  $\var{deposits}$, $\var{treasury}$, and $\var{rewards}$.
  Notice that $\var{unclaimed}$ is added to $\var{treasury}$
  and subtracted from the $\var{deposits}$.
  Moreover, $\var{refunded}$ is subtracted from $\var{deposits}$.
  (Note that $\var{deposits}-(\var{unclaimed}+\var{refunded})$
  is non-negative by Theorem~\ref{thm:non-neg-deposits}.)
  It therefore suffices to show that
  \begin{equation*}
    \begin{array}{r@{~=~}l}
    \Val(\var{rewards}\unionoverridePlus\var{refunds})
    & \Val(\var{rewards}) + \Val(\var{refunds})
    \\
    & \Val(\var{rewards}) + \var{refunded}
    \end{array}
  \end{equation*}
  But this is clear from the definition of $\unionoverridePlus$.
\end{proof}

\begin{lemma}
  \label{lemma:ru-pres-of-value}
  For every $(\Delta t,~\Delta r,~\var{rs},~\Delta f)$ in the range of $\fun{createRUpd}$,
  \begin{equation*}
    \Delta t + \Delta r + \Val(rs) + \Delta f = 0
  \end{equation*}
\end{lemma}

\begin{proof}
  In the definition of $\fun{createRUpd}$ in Figure~\ref{fig:functions:reward-update-creation},
  We see that:
  \begin{equation*}
    \begin{array}{r@{~=~}l}
      \var{rewardPot} & \var{feeSS} + \Delta r \\
      \var{R} & \var{rewardPot} - \Delta t_1 \\
      \Delta t_2 & R - \Val(\var{rs})\\
      \Delta t & \Delta t_1 + \Delta t_2 \\
    \end{array}
  \end{equation*}
  Therefore
  \begin{equation*}
    \begin{array}{r@{~=~}l}
      (\var{feeSS} + \Delta r) & \var{rewardPot} = R + \Delta t_1 = \Delta t_2 + \Val(rs) + \Delta t_1  \\
      0 & (\Delta t_1 + \Delta t_2 ) - \Delta r + \Val(rs)- \var{feeSS} \\
      0 & \Delta t - \Delta r + \Val(rs)- \var{feeSS} \\
    \end{array}
  \end{equation*}
  It then suffices to notice that $\fun{createRUpd}$ returns
  $(\Delta t,-~\Delta r,~\var{rs},~-\var{feeSS})$.
\end{proof}

\noindent

Note that Lemma~\ref{lemma:ru-pres-of-value} is not strictly need for the proof of
Theorem~\ref{thm:chain-pres-of-value}, since the $\mathsf{NEWEPOCH}$ transition
requires that $\Delta t + \Delta r + \Val(rs) + \Delta f = 0$ holds.
It does, however, give us confidence that the $\mathsf{CHAIN}$ transition can proceed.

We are now ready to prove Theorem~\ref{thm:chain-pres-of-value}.

\begin{proof}
  For a given transition $\mathsf{TR}$, let \POV{TR}
  be the statement:

  \begin{tabular}{l}
    for all environments $e$, signals $\sigma$, and states $s$, $s'$,
    $$
    $e\vdash s\trans{tr}{\sigma}s'~\implies~\Val(s) = \Val(s')$.
    $$
  \end{tabular}

  \noindent
  Our goal is to prove \POV{CHAIN}.
  Lemmas~\ref{lemma:utxo-pres-of-value} and \ref{lemma:delegs-pres-of-value} imply \POV{LEDGER},
  since $\mathsf{UTXOW}$ transforms state exactly as $\mathsf{UTXO}$ does.
  \POV{LEDGERS} then follows by straightforward induction on the length of $\Gamma$:
  the base case is trivial;
  and the inductive case follows directly from \POV{LEDGER}.
%
  \POV{SNAP} holds trivially, since it contains no value.
  Similarly, \POV{NEWPP} holds since $\var{diff}$ is added to $\var{reserves}$
  and subtracted from $\var{deposits}$.
  Therefore \POV{EPOCH} holds by Lemma~\ref{lemma:poolreap-pres-of-value}.
  \POV{MIR} holds since
  $\Val{i_{rwd}'}=\var{tot}$ in Figure~\ref{fig:rules:mir}.
  Morover, \POV{NEWEPOCH} holds in the presence of $\fun{applyRUpd}$
  since the transition requires $\Delta t + \Delta r + \Val(rs) + \Delta f = 0$.
  \POV{CHAIN} easily follows from this.
\end{proof}

\subsection{Non-negative Deposit Pot}  % TODO - this section is out of date due to no decaying deposits
\label{sec:non-negative-deposit-pot}

The \emph{deposit pot} (the variable $\var{deposits}$ in the UTxO State)
represents the amount of \emph{lovelace} that is set aside by the system as a whole for refunding deposits.
Deposits are added to this pot, which then decays exponentially over time,
and is also depleted by any refunded deposits.
At an epoch boundary, the decayed parts of any deposits (including, possibly, deposits for any transactions that will complete in future epochs)
will be distributed as additional \emph{rewards}, as described in~\cite{delegation_design}.
Since $\var{deposits}$ is only used to record the value of future refunds or rewards whose costs have
already been incurred, both it and any reward value will always be non-negative.
Note that there are two types of deposits which are recorded in the same pot: those for stake keys; and those for stake pools.
Stake keys are deregistered in the slot in which the deregistration certificates
is processed. Stake pools, however, are staged for retirement on epoch boundaries.
%
The following theorem ensures that the deposit pot is properly maintained
and will always be large enough to meet all of its obligations.


\begin{figure}[h!]
  \begin{tabular}{||l|l|l|l||}\hline\hline

    \textbf{Variable} & \textbf{Value}
                      & \textbf{Nesting Inside Chain State} & \textbf{Kind} \\ \hline
    deposits & 0 &  s.nes.es.ls.utxoSt & $\Coin$ \\ \hline
    stkCreds & $\emptyset$ & s.nes.es.ls.dpstate.dstate.stkCreds
             & $\StakeCreds$ ($\Credential\mapsto\Slot$)  \\ \hline
    stpools & $\emptyset$ & s.nes.es.ls.dpstate.pstate.stpools
            & $\StakePools$ ($\KeyHash\mapsto\Slot$)  \\ \hline
  \end{tabular}
  \caption{Initial Chain State}
  \end{figure}

\begin{theorem}[Non-negative Deposit Pot]
  \label{thm:non-neg-deposits}
  Let $n\in\N$ and $c_0\in\ChainState$ be a chain state in which $\var{deposits} ~=~0$, $\var{stkCreds}~=~\emptyset$ and $\var{stPools}~=~\emptyset$, as shown above:
%  \\~\\
  If
  \begin{equation*}
    s_0\vdash c_0\trans{\hyperref[fig:rules:chain]{chain}}{b_0}c_1,~~
    s_1\vdash c_1\trans{\hyperref[fig:rules:chain]{chain}}{b_1}c_2,~~
    \ldots,~~
    s_n\vdash c_n\trans{\hyperref[fig:rules:chain]{chain}}{b_n}c_{n+1},~~n \ge 0
  \end{equation*}
  is a sequence of valid $\mathsf{CHAIN}$ transitions,
  then $\forall i, 0 \le i \le n, \var{deposits} ~(c_{n+1}) \ge 0$.
\end{theorem}

\begin{proof}

  We will prove a slightly stronger condition, namely that some stronger invariants hold
  most of the time, and that when they do fail to hold, then $\var{deposits}$ is still non-negative.
  These stronger invariants will require a few additional definitions.
%
  Given a slot $s$, let $\ell(s)$ be the first slot of the epoch that $s$ occurs in,
  that is $\ell = \fun{firstSlot}\circ\fun{epoch}$.
  Given a mapping $m\in\mathsf{T}\to\Slot$ and a slot $s\in\Slot$,
  let $\fun{sep}$ be the function that separates $m$ into two maps,
  those whose value is strictly less than $s$ and those whose value is at least $s$.
  So,
  \begin{equation*}
    \fun{sep}~m~s = \forall x\mapsto t~\in~m,~~
    \left(\{x\mapsto t~\mid~t<s\},~\{x\mapsto t~\mid~t\geq s\}\right)
  \end{equation*}


  \noindent
  If we assume that the \emph{protocol parameters}, $pp$, are fixed\footnote{Note that the
    protocol parameters can only change in the $\mathsf{NEWPP}$ transition.}, then we can provide convenience functions
  $R_c$ and $R_p$ for the \emph{stake credential} and \emph{stake pool} refunds, respectively:
  \begin{equation*}
    \begin{array}{r@{~=~}l}
      R_c~s_0~s_1 & \refund{d_{val}}{d_{min}}{\lambda_d}{s_1-s_0} \\
      R_p~s_0~s_1 & \refund{p_{val}}{p_{min}}{\lambda_p}{s_1-s_0} \\
    \end{array}
  \end{equation*}
  where $d_{val}$, $d_{min}$, $\lambda_d$, $p_{val}$, $p_{min}$, $\lambda_p$
  are the protocol parameter values from $pp$, and $\fun{refund}$ is defined in
  Figure~\ref{fig:functions:deposits-refunds}.
  We let \DBE{c}{s} (``Deposits (precisely) Big Enough"), be the following property:
  \begin{equation}\tag{DBE}\label{DBE}
    \var{deposits}
    = \left(\sum_{\wcard\mapsto t\in C_{old}}R_c~t~\ell(s)\right)
    + |C_{new}|\cdot d_{val}
    + \left(\sum_{\wcard\mapsto t\in P_{old}}R_p~t~\ell(s)\right)
    + |P_{new}|\cdot p_{val}
  \end{equation}
  where
  \begin{equation*}
    \begin{array}{r@{~=~}l}
      C_{old},~C_{new} & \fun{sep}~\var{stkCreds}~{\ell(s)} \\
      P_{old},~P_{new} & \fun{sep}~\var{stpools}~{\ell(s)},
    \end{array}
  \end{equation*}
  for some slot, $s$, where $\var{pp}$, $\var{stkCreds}$, $\var{stpools}$ are in the corresponding chain state, $c$.
%
  In other words, \DBE{c}{s} asserts that the deposit pot is equal to the
  sum of the deposit refunds that were available at the previous epoch boundary,
  plus the sum of the initial deposit values for all the deposits from the current epoch.

  Notice that for a chain state $c$ and slot $s$, if the range of
  $\var{stkCreds}$ and $\var{stpools}$ contains only slots from the previous epoch,
  then \DBE{c}{s} is equivalent to
  \begin{equation}\tag{DEO}\label{DEO}
    \var{deposits} = \obligation{pp}{stkCreds}{stpools}{\ell(s)}
  \end{equation}
  where $\fun{obligation}$ is defined in Figure~\ref{fig:funcs:epoch-helper-rewards}.
%
  It is generally true that \DBE{c'}{s_i} holds after each subtransition of
  $s_i\vdash c_i\trans{\hyperref[fig:rules:chain]{chain}}{b_i}c_{i+1}$.
  However, this invariant can fail to hold after the
  $\hyperref[fig:delegation-transitions]{\mathsf{DELEG}}$ transition,
  since this transition can add and remove stake credentials, and can also add stake pools,
  but the deposit pot is not adjusted accordingly
  until the next subtransiton of $\hyperref[fig:rules:ledger]{\mathsf{LEDGER}}$,
  namely $\hyperref[fig:rules:utxo-shelley]{\mathsf{UTXO}}$.
%
  The invariant can also fail to hold if the slot increases while the chain state remains the same.
  That is, if \DBE{c_{i+1}}{s_i} holds, then \DBE{c_{i+1}}{s_{i+1}} can fail to hold if
  $\epoch{s_i} < \epoch{s_{i+1}}$, since the value of the deposit
  in the left hand side of equation~\ref{DBE} remains the same, but the
  refunded values become smaller\footnote{Note that if $\epoch{s_i} = \epoch{s_{i+1}}$, then \DBE{c_{i+1}}{s_{i+1}} is trivially true.}.
  Therefore, in this situation we can consider the slightly weaker constraint:
  \begin{equation}\tag{DGO}\label{DGTO}
    \var{deposits} \geq \obligation{pp}{stkCreds}{stpools}{\ell(s)}
  \end{equation}
  The difference between the left and right hand sides of the inequality
  corresponds to the lovelace value in $c_{i+1}$ that decays between $s_i$ and $s_{i+1}$.

  There are four sub-transitions where $\var{deposits}$ is changed:
  $\mathsf{SNAP}$ (Figure~\ref{fig:rules:snapshot}),
  $\mathsf{POOLREAP}$ (Figure~\ref{fig:rules:pool-reap}),
  $\mathsf{NEWPP}$ (Figure~\ref{fig:rules:new-proto-param}),
  $\mathsf{UTXO}$ (Figure~\ref{fig:rules:utxo-shelley}).
  This ordering is also the order in which $\var{deposits}$ is changed.
  Of these sub-transitions, only $\mathsf{UTXO}$ actually changes the value of $\var{deposits}$
  when $s_i$ is in the same epoch as $s_i$.
  (We say that $s_i$ \emph{crosses the epoch boundary} if the precondition of
  Rule~\ref{eq:new-epoch} in Figure~\ref{fig:rules:new-epoch} is met,
  namely if $\epoch{s_i} \ge e_\ell+1$.)
%
  The proof then proceeds by induction on $n$, showing the following:
  \begin{itemize}
    \item
      Let $c$ be the chain state after the $\mathsf{SNAP}$ transition
      in $s_i\vdash c_i\trans{\hyperref[fig:rules:chain]{chain}}{b_i}c_{i+1}$.
      If \DGO{c_i}{s_i}, then \DBE{c}{s_i} holds.
    \item $\mathsf{POOLREAP}$ preserves \ref{DBE}.
    \item $\mathsf{NEWPP}$ preserves \ref{DBE}.
    \item The property for $\mathsf{UTXO}$ requires a bit of explanation.
      Let $\var{nes}\in\NewEpochState$ be the new epoch state in $c_i$.
      Note that the property \ref{DBE} makes sense for values of $\NewEpochState$
      since it contains all the relevant variables.
      Similarly, \ref{DBE} also makes sense for values of $\UTxOState\times\PParams$.
      Let
      $$
        {\begin{array}{c}
           \var{gkeys} \\
         \end{array}}
        \vdash\var{nes}\trans{\hyperref[fig:rules:tick]{tick}}{\var{bh}}\var{nes'}
      $$
      be the first sub-transition of
      $s_i\vdash c_i\trans{\hyperref[fig:rules:chain]{chain}}{b_i}c_{i+1}$.
      If \DBE{\var{nes'}}{s_i} holds, then \DBE{(us', pp)}{s_i} holds for every transaction
      $tx$ in $b_i$, where:
      $$
      \var{env}\vdash \var{us} \trans{\hyperref[fig:rules:utxow-shelley]{utxo}}{tx} \var{us'},
      $$
      is a sub-transition of
      $s_i\vdash c_i\trans{\hyperref[fig:rules:chain]{chain}}{b_i}c_{i+1}$,
      and $\var{pp}$ is the protocol parameters in $\var{nes'}$.
  \end{itemize}

  \noindent
  Case $\hyperref[fig:rules:snapshot]{\mathsf{SNAP}}$.
  We must show that
  if $c$ is the chain state after the $\mathsf{SNAP}$ transition
  in $s_i\vdash c_i\trans{\hyperref[fig:rules:chain]{chain}}{b_i}c_{i+1}$,
  and \DGO{c_i}{s_i} holds, then so does \DBE{c}{s_i}.
%
  We can assume that $s_i$ crosses the epoch boundary,
  since otherwise the $\mathsf{SNAP}$ transition will not occur.
  Since the $\mathsf{SNAP}$ transition only happens within the $\mathsf{TICK}$ transition
  on the epoch boundary, it follows that
  $c_i$ does not contain any stake credentials or pools from the current epoch,
  and so \ref{DBE} will be equivalent to \ref{DEO} (the current epoch is $\epoch{s_i}$).
  However, \DBE{c}{s_i} holds trivially, since it is determined from the $\fun{obligation}$ value.
  \\~\\
  Case $\hyperref[fig:rules:pool-reap]{\mathsf{POOLREAP}}$.
  We must show that \ref{DBE} is preserved.
%
  We again assume that $s_i$ crosses the epoch boundary.
  The $\mathsf{POOLREAP}$ transition does the following:
  \begin{enumerate}
    \item leaves $\var{stkCreds}$ unchanged,
    \item removes $\var{retired}$ from $\var{stpools}$,
    \item subtracts $\var{unclaimed}+\var{refunded}$ from $\var{deposits}$.
  \end{enumerate}
%
  Notice that the domain of the $\var{pr}$ is $\var{retired}$,
  and similarly the domain of the $\var{rewardAcnts}$ is also $\var{retired}$
  since the domains of $\var{stpools}$ and $\var{poolParams}$ are the same.
  Therefore $\var{retired}$ is the disjoint union of
  $\dom({\var{refunds}})$ and $\dom({\var{mRefunds}})$, so that
  \begin{equation*}
    \begin{array}{r@{~=~}l}
      \var{unclaimed}+\var{refunded}
      &
      \left(
        \sum\limits_{\wcard\mapsto t\in\var{refunds}}R_p~t~\ell(s)
      \right)+
      \left(
        \sum\limits_{\wcard\mapsto t\in\var{mRefunds}}R_p~t~\ell(s)
      \right)
      \\
      &
      \sum\limits_{\wcard\mapsto t\in\var{rewardAcnts'}}R_p~t~\ell(s)
      \\
      &
      \left(
        \sum\limits_{\wcard\mapsto t\in\var{stpools}}R_p~t~\ell(s)
      \right)-
      \left(
        \sum\limits_{\wcard\mapsto t\in\var{retired}\subtractdom\var{stpools}}R_p~t~\ell(s)
      \right)
    \end{array}
  \end{equation*}
  Therefore, it follows that if \ref{DEO} holds before $\mathsf{POOLREAP}$, then it also holds afterwards.
  \\~\\
  Case $\hyperref[fig:rules:new-proto-param]{\mathsf{NEWPP}}$.
  We must show that \ref{DBE} is preserved.
%
  We again assume that $s_i$ crosses the epoch boundary.
  In this transition $\var{pp}$ can change, but $\var{stkCreds}$, $\var{stpools}$,
  and $\var{deposits}$ do not change.
  As in the $\mathsf{SNAP}$ case, \DBE{c}{s_i} holds trivially,
  since it is set to the value that is determined by $\fun{obligation}$.
  \\~\\
  Case $\hyperref[fig:rules:utxo-shelley]{\mathsf{UTXO}}$.
  We assume that \DBE{\var{nes'}}{s_i} holds, where $\var{nes'}$
  is the new epoch state after the $\mathsf{TICK}$ transition.
  We must show that \ref{DBE} is preserved after each $\mathsf{UTXO}$ transition.
%
  The $\mathsf{DELEGS}$ transition can result in values being
  added to or deleted from $\var{stkCreds}$, and added to $\var{stpools}$.
  Let $A_s$ be the added stake credentials, $D_s$ be the deleted credentials, and
  $A_p$ be the added stake pools, where $\var{stkCreds}'$ is the stake credential mapping
   $\var{stpools}'$ is the stake pools, and $\var{deposits}'$ is the deposit pot  after $\mathsf{DELEGS}$.
  We have that
  \begin{equation*}
    \begin{array}{rcl}
      \var{D_s} & \subseteq & \var{\var{stkCreds}\cup\var{A_s}} \\
      \var{stkCreds}' & = & (\var{stkCreds}\cup\var{A_s})\setminus\var{D_s} \\
      \var{stpools}' & = & \var{stpools}\cup\var{A_p} \\
    \end{array}
  \end{equation*}
  The slots in the range of $A_s$ will all be equal to $s_i$,
  but the slots in the range of $D_s$
may either be from the current epoch or an earlier one, so we split them using $\fun{sep}$:
  \begin{equation*}
    (\var{D_{s\_old}},~\var{D_{s\_new}}) = \fun{sep}~\var{D_s}~\ell(s_i)
  \end{equation*}
  We must then show that
  \begin{equation*}
    \var{deposits}' = \var{deposits}
    + |A_s|\cdot d_{val}
    + |P_c|\cdot p_{val}
    - |D_{s\_new}|\cdot d_{val}
    - \left(\sum_{\wcard\mapsto t\in D_{s\_old}}R_c~t~\ell(s_i)\right)
  \end{equation*}
  Looking at the $\mathsf{UTXO}$ transition in Figure~\ref{fig:rules:utxo-shelley},
  \begin{equation*}
    \var{deposits}' = \var{deposits} + \totalDeposits{pp}{stpools}{(\txcerts{tx})}
    - (\var{refunded} + \var{decayed})
  \end{equation*}
  The function $\fun{totalDeposits}$ is defined in Figure~\ref{fig:functions:deposits-refunds}
  and it is clear that here it is equal to
  $$|A_s|\cdot d_{val} + |P_c|\cdot p_{val.}$$
  Recall that
  \begin{equation*}
    \begin{array}{r@{~=~}l}
      \var{refunded} & \keyRefunds{pp}{stkCreds}~{tx} \\
      \var{decayed} & \decayedTx{pp}{stkCreds}~{tx}
    \end{array}
  \end{equation*}
  where $\fun{keyRefunds}$ is defined in Figure~\ref{fig:functions:deposits-refunds}.
  This iterates $\fun{keyRefund}$ from the same figure,
  which in turn just looks up the creation slot for a transaction and returns $R_c$.
  The function to calculate the value of decayed deposits, $\fun{decayedTx}$, is defined in Figure~\ref{fig:functions:deposits-decay}.
  This iterates $\fun{decayedKey}$ from the same figure.
  Therefore, to show that
  \begin{equation}\label{deleted-is-refunds-plus-decayed}
    |D_{s\_new}|\cdot d_{val} + \sum_{\wcard\mapsto t\in D_{s\_old}}R_c~t~\ell(s_i)
    = \var{refunded} + \var{decayed},
  \end{equation}
  and thus complete the proof for the $\mathsf{UTXO}$ case,
  it suffices to show that for a given $\var{c}\mapsto s\in D_s$,
  the $R_c$ value plus the $\fun{decayedKey}$ value that is associated with the stake
  credential $c$ is equal to $d_{val}$ if $\epoch(s)=\epoch(s_i)$, and is otherwise equal to $R_c~s~\ell(s_i)$.
  Looking at the definition of $\fun{decayedKey}$, observe that if $\epoch(s)=\epoch(s_i)$
  then $\var{start}=\var{created}$ and so the decayed value is $(R_c~s~s)-(R_c~s~s_i)$.
  However, $R_c~s~s = d_{val}$, so the refund plus the decayed value is
  $d_{val}-(R_c~s~s_i)+(R_c~s~s_i)=d_{val}$.
  Otherwise, if $s$ is from a previous epoch, then $\var{start}=\ell(s_i)$, and so
  the decayed value is $(R_c~s~\ell(s_i))-(R_c~s~s_i)$.
  The refund plus the decayed value is thus
  $(R_c~s~\ell(s_i))-(R_c~s~s_i)+(R_c~s~s_i)=(R_c~s~\ell(s_i))$.
  Therefore, equation~\ref{deleted-is-refunds-plus-decayed} holds, and
  consequently so also does \DBE{c'}{s_i}.

\end{proof}



\subsection{Header-Only Validation}
\label{sec:header-only-validation}
The header-only validation properties of the Shelley Ledger are the analogs
of those from Section 8.1 of \cite{byron_chain_spec}.

In any given chain state, the consensus layer needs to be able to validate the
block headers without having to download the block bodies.
Property~\ref{prop:header-only-validation} states that if an extension of a
chain that spans less than $\StabilityWindow$ slots is valid, then validating the
headers of that extension is also valid. This property is useful for its
converse: if the header validation check for a sequence of headers does not
pass, then we know that the block validation that corresponds to those headers
will not pass either.

First we define the header-only version of the $\mathsf{CHAIN}$ transition,
which we call $\mathsf{CHAINHEAD}$.
It is very similar to $\mathsf{CHAIN}$, the differences being:
\begin{itemize}
  \item The $\mathsf{CHAINHEAD}$ signal is not a block, but a block header ($\BHeader$).
  \item $\mathsf{CHAINHEAD}$ does not call $\mathsf{BBODY}$.
  \item $\mathsf{CHAINHEAD}$ does not call $\mathsf{TICK}$, but instead
    calls the similar $\mathsf{TICKF}$, which differs by
    not calling the reward update transition $\mathsf{RUPD}$.
  \item $\mathsf{CHAINHEAD}$ does not store the new epoch state $\var{nes}$ in
    its state, but rather contains it in the environment.
    We will conveniently \textbf{abuse the tuple notation} and write
    $(nes, \tilde{s}) = s$
    for splitting the chain state into the new epoch state and the remaiing fields.
\end{itemize}

\begin{figure}[ht]
  \begin{equation}\label{eq:tickf}
    \inference[TickForecast]
    {
      {
        \vdash
        \var{nes}
        \trans{\hyperref[fig:rules:new-epoch]{newepoch}}{\epoch{slot}}
        \var{nes}'
      }
      \\~\\
      (\var{e_\ell'},~\var{b_{prev}'},~\var{b_{cur}'},~\var{es'},~\var{ru'},~\var{pd'})
      \leteq\var{nes'}
      \\
      \var{es''}\leteq\fun{adoptGenesisDelegs}~\var{es'}~\var{slot}
      \\
      \var{forecast}\leteq
      (\var{e_\ell'},~\var{b_{prev}'},~\var{b_{cur}'},~\var{es''},~\var{ru'},~\var{pd'})
      \\~\\
    }
    {
      \vdash\var{nes}\trans{tickf}{\var{slot}}\varUpdate{\var{forecast}}
    }
  \end{equation}
  \caption{Tick Forecast rules}
  \label{fig:rules:tickf}
\end{figure}

\clearpage

\begin{figure}[ht]
  \begin{equation}\label{eq:chain-head}
    \inference[ChainHead]
    {
      \var{bhb} \leteq \bhbody{bh}
      &
      \var{s} \leteq \bslot{bhb}
      \\~\\
      \fun{prtlSeqChecks}~\var{lab}~\var{bh}
      \\~\\
      {
        \vdash\var{nes}\trans{\hyperref[fig:rules:tickf]{tickf}}{\var{s}}\var{forecast}
      } \\~\\
      (\var{e_1},~\wcard,~\wcard,~\wcard,~\wcard,~\wcard)
        \leteq\var{nes} \\
      (\var{e_2},~\wcard,~\wcard,~\var{es},~\wcard,~\wcard,~\var{pd})
        \leteq\var{forecast} \\
        (\var{acnt},~\wcard,\var{ls},~\wcard,~\var{pp})\leteq\var{es}\\
        ( \wcard,
          ( (\wcard,~\wcard,~\wcard,~\wcard,~\var{genDelegs},~\wcard),~
          (\wcard,~\wcard,~\wcard)))\leteq\var{ls}\\
          \var{ne} \leteq  \var{e_1} \neq \var{e_2}\\
          \eta_{ph} \leteq \prevHashToNonce{(\lastAppliedHash{lab})} \\~\\
      \fun{chainChecks}~
        \MaxMajorPV~(\fun{maxHeaderSize}~\var{pp},~\fun{maxBlockSize}~\var{pp},~\fun{pv}~\var{pp})~
        \var{bh}\\~\\
      {
        {\begin{array}{c}
        \var{pp} \\
        \eta_c \\
        \eta_\var{ph} \\
        \end{array}}
        \vdash
        {\left(\begin{array}{c}
        \eta_0 \\
        \eta_h \\
        \end{array}\right)}
        \trans{\hyperref[fig:rules:tick-nonce]{tickn}}{\var{ne}}
        {\left(\begin{array}{c}
        \eta_0' \\
        \eta_h' \\
        \end{array}\right)}
      }\\~\\~\\
      {
        {\begin{array}{c}
            (\fun{d}~\var{pp}) \\
            \var{pd} \\
            \var{genDelegs} \\
            \eta_0' \\
         \end{array}}
        \vdash
        {\left(\begin{array}{c}
              \var{cs} \\
              \eta_v \\
              \eta_c \\
        \end{array}\right)}
        \trans{\hyperref[fig:rules:prtcl]{prtcl}}{\var{bh}}
        {\left(\begin{array}{c}
              \var{cs'} \\
              \eta_v' \\
              \eta_c' \\
        \end{array}\right)}
      } \\~\\~\\
      \var{lab'}\leteq (\bblockno{bhb},~\var{s},~\bhash{bh} ) \\
    }
    {
      \var{nes}
      \vdash
      {\left(\begin{array}{c}
            \var{cs} \\
            \eta_0 \\
            \eta_v \\
            \eta_c \\
            \eta_h \\
            \var{lab} \\
      \end{array}\right)}
      \trans{chainhead}{\var{bh}}
      {\left(\begin{array}{c}
            \varUpdate{\var{cs}'} \\
            \varUpdate{\eta_0'} \\
            \varUpdate{\eta_v'} \\
            \varUpdate{\eta_c'} \\
            \varUpdate{\eta_h'} \\
            \varUpdate{\var{lab}'} \\
      \end{array}\right)}
    }
  \end{equation}
  \caption{Chain-Head rules}
  \label{fig:rules:chainhead}
\end{figure}

\begin{property}[Header only validation]\label{prop:header-only-validation}
  For all states $s$ with slot number $t$\footnote{i.e. the
    component $\var{s_\ell}$ of the last applied block of $s$ equals $t$},
    and chain extensions $E$ with corresponding headers $H$ such that:
  %
  $$
  0 \leq t_E - t  \leq \StabilityWindow
  $$
  %
  we have:
  %
  $$
  \vdash s \transtar{\hyperref[fig:rules:chain]{chain}}{E} s'
  \implies
  \var{nes} \vdash \tilde{s} \transtar{\hyperref[fig:rules:chainhead]{chainhead}}{H} s''
  $$
  where $s=(\var{nes},~\tilde{s})$,
  $t_E$ is the maximum slot number appearing in the blocks contained in
  $E$, and $H$ is obtained from $E$ by applying $\fun{bheader}$ to each block in $E$.
\end{property}

\begin{property}[Body only validation]\label{prop:body-only-validation}
  For all states $s$ with slot number $t$, and chain
  extensions $E = [b_0, \ldots, b_n]$ with corresponding headers $H$ such that:
  $$
  0 \leq t_E - t  \leq \StabilityWindow
  $$
  we have that for all $i \in [1, n]$:
  $$
  \var{nes} \vdash \tilde{s} \transtar{\hyperref[fig:rules:chainhead]{chainhead}}{H} s_{h}
  \wedge
  \vdash (\var{nes},~\tilde{s}) \transtar{\hyperref[fig:rules:chain]{chain}}{[b_0 \ldots b_{i-1}]} s_{i-1}
  \implies
  \var{nes'} \vdash \tilde{s}_{i-1}\trans{\hyperref[fig:rules:chainhead]{chainhead}}{h_i} s'_{h}
  $$
  where $s_{i-1}=(\var{nes'},~\tilde{s}_{i-1})$,
  $t_E$ is the maximum slot number appearing in the blocks contained in $E$.
\end{property}

Property~\ref{prop:body-only-validation} states that if we validate a sequence
of headers, we can validate their bodies independently and be sure that the
blocks will pass the chain validation rule. To see this, given an environment
$e$ and initial state $s$, assume that a sequence of headers
$H = [h_0, \ldots, h_n]$ corresponding to blocks in $E = [b_0, \ldots, b_n]$ is
valid according to the $\mathsf{chainhead}$ transition system:
%
$$
\var{nes} \vdash \tilde{s} \transtar{\hyperref[fig:rules:chainhead]{chainhead}}{H} \tilde{s'}
$$
%
Assume the bodies of $E$ are valid
according to the $\mathsf{bbody}$ rules, but $E$ is not valid according to
the $\mathsf{chain}$ rule. Assume that there is a $b_j \in E$ such that it is
\textbf{the first block} such that does not pass the $\mathsf{chain}$
validation. Then:
%
$$
\vdash (\var{nes},~\tilde{s}) \transtar{\hyperref[fig:rules:chain]{chain}}{[b_0, \ldots b_{j-1}]} s_j
$$
But by Property~\ref{prop:body-only-validation} we know that
%
$$
\var{nes}_j \vdash \tilde{s}_j \trans{\hyperref[fig:rules:chainhead]{chainhead}}{h_j} \tilde{s}_{j+1}
$$
which means that block $b_j$ has valid headers, and this in turn means that the
validation of $b_j$ according to the chain rules must have failed because it
contained an invalid block body. But this contradicts our assumption that the
block bodies were valid.

\begin{property}[Existence of roll back function]\label{prop:roll-back-funk}
  There exists a function $\fun{f}$ such that for all chains
  $$C = C_0 ; b; C_1$$
  we have that if for all alternative chains $C'_1$, $\size{C'_1} \leq \frac{\StabilityWindow}{2}$, with
  corresponding headers $H'_1$
  $$
  \vdash s_0 \transtar{\hyperref[fig:rules:chain]{chain}}{C_0;b} s_1 \transtar{\hyperref[fig:rules:chain]{chain}}{C_1} s_2
  \wedge
  \vdash s_1 \transtar{\hyperref[fig:rules:chain]{chain}}{C_1'} s'_1
  \implies
  \var{nes} \vdash \tilde{s} \transtar{\hyperref[fig:rules:chainhead]{chainhead}}{H'_1} s_h
  $$
\end{property}
where $\fun{f}~(\bheader{b})~s_2=(\var{nes},~\tilde{s})$.

Property~\ref{prop:roll-back-funk} expresses the fact the there is a function
that allow us to recover the header-only state by rolling back at most $k$
blocks, and use this state to validate the headers of an alternate chain. Note
that this property is not inherent to the $\mathsf{chain}$ rules and can be
trivially satisfied by any function that keeps track of the history of the
intermediate chain states up to $k$ blocks back. This property is stated here
so that it can be used as a reference for the tests in the consensus layer,
which uses the rules presented in this document.


\subsection{Validity of a Ledger State}
\label{sec:valid-ledg-state}

Many properties only make sense when applied to a valid ledger state. In
informal terms, a valid ledger state $l$ can only be reached when starting from
an initial state $l_{0}$ (ledger in the genesis state) and only executing LEDGER
state transition rules as specified in Section~\ref{sec:ledger-trans} which
changes either the UTxO or the delegation state.

\begin{figure}[ht]
  \centering
  \begin{align*}
    \genesisId & \in & \TxId \\
    \genesisTxOut & \in & \TxOut \\
    \genesisUTxO & \coloneqq & (\genesisId, 0) \mapsto \genesisTxOut
    \\
    \ledgerState & \in & \left(
                         \begin{array}{c}
                           \UTxOState \\
                           \DPState
                         \end{array}
    \right)\\
               && \\
    \fun{getUTxO} & \in & \UTxOState \to \UTxO \\
    \fun{getUTxO} & \coloneqq & (\var{utxo}, \wcard, \wcard, \wcard) \mapsto \var{utxo}
  \end{align*}
  \caption{Definitions and Functions for Valid Ledger State}
  \label{fig:valid-ledger}
\end{figure}

In Figure~\ref{fig:valid-ledger} \genesisId{} marks the transaction identifier
of the initial coin distribution, where \genesisTxOut{} represents the initial
UTxO. It should be noted that no corresponding inputs exists, i.e., the
transaction inputs are the empty set for the initial transaction. The function
\fun{getUTxO} extracts the UTxO from a UTxO state.

\begin{definition}[\textbf{Valid Ledger State}]
  \begin{multline*}
    \forall l_{0},\ldots,l_{n} \in \LState, lenv_{0},\ldots,lenv_{n} \in \LEnv,
    l_{0} = \left(
      \begin{array}{c}
        \genesisUTxOState \\
        \left(
        \begin{array}{c}
          \emptyset\\
          \emptyset
        \end{array}
        \right)
      \end{array}
    \right)  \\
    \implies \forall 0 < i \leq n, (\exists tx_{i} \in \Tx,
    lenv_{i-1}\vdash l_{i-1} \trans{ledger}{tx_{i}} l_{i}) \implies
    \applyFun{validLedgerState} l_{n}
  \end{multline*}
  \label{def:valid-ledger-state}
\end{definition}

Definition~\ref{def:valid-ledger-state} defines a valid ledger state reachable
from the genesis state via valid LEDGER STS transitions. This gives a
constructive rule how to reach a valid ledger state.

\subsection{Ledger Properties}
\label{sec:ledger-properties}

The following properties state the desired features of updating a valid ledger
state.

\begin{property}[\textbf{Preserve Balance}]
  \begin{multline*}
    \forall \var{l}, \var{l'} \in \LState: \applyFun{validLedgerstate}{l},
    l=(u,\wcard,\wcard,\wcard), l' = (u',\wcard,\wcard,\wcard)\\
    \implies \forall \var{tx} \in \Tx, lenv \in\LEnv, lenv \vdash\var{u} \trans{utxow}{tx} \var{u'} \\
    \implies \applyFun{destroyed}{pc~utxo~stkCreds~rewards~tx} =
    \applyFun{created}{pc~stPools~tx}
  \end{multline*}
  \label{prop:ledger-properties-1}
\end{property}

Property~\ref{prop:ledger-properties-1} states that for each valid ledger $l$,
if a transaction $tx$ is added to the ledger via the state transition rule UTXOW
to the new ledger state $l'$, the balance of the UTxOs in $l$ equals the balance
of the UTxOs in $l'$ in the sense that the amount of created value in $l'$
equals the amount of destroyed value in $l$. This means that the total amount of
value is left unchanged by a transaction.

\begin{property}[\textbf{Preserve Balance Restricted to TxIns in Balance of
    TxOuts}]
  \begin{multline*}
    \forall \var{l}, \var{l'} \in \ledgerState: \applyFun{validLedgerstate}{l},
    l=(u,\wcard,\wcard,\wcard), l' = (u',\wcard,\wcard,\wcard)\\
    \implies \forall \var{tx} \in \Tx, lenv \in\LEnv, lenv \vdash \var{u}
    \trans{utxow}{tx} \var{u'} \\
    \implies \fun{ubalance}(\applyFun{txins}{tx} \restrictdom
    \applyFun{getUTxO}{u}) = \fun{ubalance}(\applyFun{outs}{tx}) +
    \applyFun{txfee}{tx} + depositChange
  \end{multline*}
  \label{prop:ledger-properties-2}
\end{property}

Property~\ref{prop:ledger-properties-2} states a slightly more detailed relation
of the balances change. For ledgers $l, l'$ and a transaction $tx$ as above, the
balance of the UTxOs of $l$ restricted to those whose domain is in the set of
transaction inputs of $tx$ equals the balance of the transaction outputs of $tx$
minus the transaction fees and the change in the deposit
$depositChange$~(cf.~Fig.~\ref{fig:rules:utxo-shelley}).

\begin{property}[\textbf{Preserve Outputs of Transaction}]
  \begin{multline*}
    \forall \var{l}, \var{l'} \in \ledgerState: \applyFun{validLedgerstate}{l},
    l=(u,\wcard,\wcard,\wcard), l' = (u',\wcard,\wcard,\wcard)\\
    \implies \forall \var{tx} \in \Tx, lenv \in\LEnv, lenv \vdash \var{u}
    \trans{utxow}{tx} \var{u'} \implies \forall \var{out} \in
    \applyFun{outs}{tx}, out \in \applyFun{getUTxO}{u'}
  \end{multline*}
  \label{prop:ledger-properties-3}
\end{property}

Property~\ref{prop:ledger-properties-3} states that for all ledger states
$l, l'$ and transaction $tx$ as above, all output UTxOs of $tx$ are in the UTxO
set of $l'$, i.e., they are now available as unspent transaction output.

\begin{property}[\textbf{Eliminate Inputs of Transaction}]
  \begin{multline*}
    \forall \var{l}, \var{l'} \in \ledgerState: \applyFun{validLedgerstate}{l},
    l=(u,\wcard,\wcard,\wcard), l' = (u',\wcard,\wcard,\wcard)\\
    \implies \forall \var{tx} \in \Tx, lenv \in\LEnv, lenv \vdash \var{u}
    \trans{utxow}{tx} \var{u'} \implies \forall \var{in} \in
    \applyFun{txins}{tx}, in \not\in \fun{dom}(\applyFun{getUTxO}{u'})
  \end{multline*}
  \label{prop:ledger-properties-4}
\end{property}

Property~\ref{prop:ledger-properties-4} states that for all ledger states
$l, l'$ and transaction $tx$ as above, all transaction inputs $in$ of $tx$ are
not in the domain of the UTxO of $l'$, i.e., these are no longer available to
spend.

\begin{property}[\textbf{Completeness and Collision-Freeness of new Transaction
    Ids}]
  \begin{multline*}
    \forall \var{l}, \var{l'} \in \ledgerState: \applyFun{validLedgerstate}{l},
    l=(u,\wcard,\wcard,\wcard), l' = (u',\wcard,\wcard,\wcard)\\
    \implies \forall \var{tx} \in \Tx, lenv \in\LEnv, lenv \vdash \var{u}
    \trans{utxow}{tx} \var{u'} \\ \implies \forall ((txId', \wcard) \mapsto
    \wcard) \in \applyFun{outs}{tx}, ((txId, \wcard) \mapsto \wcard)
    \in\applyFun{getUTxO}{u} \implies \var{txId'} \neq \var{txId}
  \end{multline*}
  \label{prop:ledger-properties-5}
\end{property}

Property~\ref{prop:ledger-properties-5} states that for ledger states $l, l'$
and a transaction $tx$ as above, the UTxOs of $l'$ contain all newly created
UTxOs and the referred transaction id of each new UTxO is not used in the UTxO
set of $l$.

\begin{property}[\textbf{Absence of Double-Spend}]
  \begin{multline*}
    \forall l_{0},\ldots,l_{n} \in \ledgerState, l_{0} =
    \left(
      \begin{array}{c}
        \left\{
        \genesisUTxO
        \right\} \\
        \left(
        \begin{array}{c}
          \emptyset\\
          \emptyset
        \end{array}
        \right)
      \end{array}
    \right) \wedge \applyFun{validLedgerState} l_{n}, l_{i}=(u_{i},\wcard,\wcard,\wcard)\\
    \implies \forall 0 < i \leq n, tx_{i} \in \Tx, lenv_{i}\in\LEnv,
    lenv_{i} \vdash u_{i-1}
    \trans{ledger}{tx_{i}} u_{i} \wedge \applyFun{validLedgerState} l_{i} \\
    \implies \forall j < i, \applyFun{txins}{tx_{j}} \cap
    \applyFun{txins}{tx_{i}} = \emptyset
  \end{multline*}
  \label{prop:ledger-properties-no-double-spend}
\end{property}

Property~\ref{prop:ledger-properties-no-double-spend} states that for each valid
ledger state $l_{n}$ reachable from the genesis state, each transaction $t_{i}$
does not share any input with any previous transaction $t_{j}$. This means that
each output of a transition is spent at most once.

\subsection{Ledger State Properties for Delegation Transitions}
\label{sec:ledg-prop-deleg}

\begin{figure}[ht]
  \centering
  \begin{align*}
    \fun{getStDelegs} & \in & \DState \to \powerset \Credential \\
    \fun{getStDelegs} & \coloneqq &
                                    (\var{stkCreds}, \wcard,
                                    \wcard,\wcard,\wcard,\wcard) \to \var{stkCreds} \\
                      &&\\
    \fun{getRewards} & \in & \DState \to (\AddrRWD \mapsto \Coin) \\
    \fun{getRewards} & \coloneqq & (\wcard, \var{rewards},
                                   \wcard,\wcard,\wcard,\wcard)
                                   \to \var{rewards} \\
                      &&\\
    \fun{getDelegations} & \in & \DState \to (\Credential \mapsto \KeyHash) \\
    \fun{getDelegations} & \coloneqq & (\wcard, \wcard,
                                       \var{delegations},\wcard,\wcard,\wcard) \to
                                       \var{delegations} \\
                      &&\\
    \fun{getStPools} & \in & \LState \to (\KeyHash \mapsto \DCertRegPool) \\
    \fun{getStPools} & \coloneqq & (\wcard, (\wcard,
                                   (\var{stpools},\wcard,\wcard,\wcard))) \to \var{stpools} \\
                      &&\\
    \fun{getRetiring} & \in & \LState \to (\KeyHash \mapsto \Epoch) \\
    \fun{getRetiring} & \coloneqq & (\wcard, (\wcard,
                                    (\wcard, \wcard, \var{retiring},\wcard))) \to \var{retiring} \\
  \end{align*}
  \caption{Definitions and Functions for Stake Delegation in Ledger States}
  \label{fig:stake-delegation-functions}
\end{figure}


\begin{property}[\textbf{Registered Staking Credential with Zero Rewards}]
  \begin{multline*}
    \forall \var{l}, \var{l'} \in \ledgerState: \applyFun{validLedgerstate}{l},
    l = (\wcard, ((d, \wcard), \wcard)), l' = (\wcard, ((d',\wcard), \wcard)), dEnv\in\DEnv \\
    \implies \forall \var{c} \in \DCertRegKey, dEnv\vdash \var{d}
    \trans{deleg}{c} \var{d'} \implies \applyFun{cwitness}{c} = \var{hk}\\
    \implies hk\not\in \fun{getStDelegs}~\var{d} \implies \var{hk} \in
    \applyFun{getStDelegs}{d'} \wedge
    (\applyFun{getRewards}\var{d'})[\fun{addr_{rwd}}{hk}] = 0
  \end{multline*}
  \label{prop:ledger-properties-6}
\end{property}

Property~\ref{prop:ledger-properties-6} states that for each valid ledger state
$l$, if a delegation transaction of type $\DCertRegKey$ is executed, then in the
resulting ledger state $l'$, the set of staking credential of $l'$ includes the
credential $hk$ associated with the key registration certificate and the
associated reward is set to 0 in $l'$.

\begin{property}[\textbf{Deregistered Staking Credential}]
  \begin{multline*}
    \forall \var{l}, \var{l'} \in \ledgerState: \applyFun{validLedgerstate}{l},
    l = (\wcard, (d, \wcard)), l' = (\wcard, (d', \wcard)), dEnv\in\DEnv \\
    \implies \forall \var{c} \in \DCertDeRegKey, dEnv\vdash\var{d}
    \trans{deleg}{c} \var{d'} \implies \applyFun{cwitness}{c} = \var{hk}\\
    \implies \var{hk} \not\in \applyFun{getStDelegs}{d'} \wedge hk\not\in
    \left\{ \fun{stakeCred_{r}}~sc\vert
      sc\in\fun{dom}(\applyFun{getRewards}{d'})
    \right\}\\
    \wedge hk \not\in \fun{dom}(\applyFun{getDelegations}{d'}))
  \end{multline*}
  \label{prop:ledger-properties-7}
\end{property}

Property~\ref{prop:ledger-properties-7} states that for $l, l'$ as above but
with a delegation transition of type $\DCertDeRegKey$, the staking credential
$hk$ associated with the deregistration certificate is not in the set of staking
credentials of $l'$ and is not in the domain of either the rewards or the
delegation map of $l'$.

\begin{property}[\textbf{Delegated Stake}]
  \begin{multline*}
    \forall \var{l}, \var{l'} \in \ledgerState: \applyFun{validLedgerstate}{l},
    l = (\wcard, (d,\wcard)), l' = (\wcard, (d',\wcard)), dEnv\in\DEnv \\
    \implies \forall \var{c} \in \DCertDeleg, dEnv \vdash\var{d}
    \trans{deleg}{c} \var{d'} \implies \applyFun{cwitness}{c} = \var{hk}\\
    \implies \var{hk} \in \applyFun{getStDelegs}{d'} \wedge
    (\applyFun{getDelegations}{d'})[hk] = \applyFun{dpool}{c}
  \end{multline*}
  \label{prop:ledger-properties-8}
\end{property}

Property~\ref{prop:ledger-properties-8} states that for $l, l'$ as above but
with a delegation transition of type $\DCertDeleg$, the staking credential $hk$
associated with the deregistration certificate is in the set of staking
credentials of $l$ and delegates to the staking pool associated with the
delegation certificate in $l'$.

\begin{property}[\textbf{Genesis Keys are Always All Delegated}]
  \label{prop:genkeys-delegated}
  \begin{multline*}
    \forall \var{l}, \var{l'} \in \LState: \applyFun{validLedgerstate}{l},\\
    \implies \forall \Gamma \in \seqof{\Tx}, env \in (\Slot \times \PParams), \\
    env \vdash\var{l} \trans{ledgers}{\Gamma} \var{l'} \implies |genDelegs| = 7
  \end{multline*}
\end{property}

Property \ref{prop:genkeys-delegated} states that all seven of the genesis keys
are constantly all delegated after applying a list of transactions to a valid ledger
state.

\subsection{Ledger State Properties for Staking Pool Transitions}
\label{sec:ledg-state-prop}

\begin{property}[\textbf{Registered Staking Pool}]
  \begin{multline*}
    \forall \var{l}, \var{l'} \in \ledgerState: \applyFun{validLedgerstate}{l},
    l = (\wcard, (\wcard, p)), l' = (\wcard, (\wcard, p')), pEnv\in\PEnv \\
    \implies \forall \var{c} \in \DCertRegPool, \var{p} \trans{pool}{c} \var{p'}
    \implies \applyFun{cwitness}{c} = \var{hk}\\ \implies
    \var{hk}\in\applyFun{getStPools}{p'} \wedge \var{hk} \not\in
    \applyFun{getRetiring}{p'}
  \end{multline*}
  \label{prop:ledger-properties-9}
\end{property}

Property~\ref{prop:ledger-properties-9} states that for $l, l'$ as above but
with a delegation transition of type $\DCertRegPool$, the key $hk$ is associated
with the author of the pool registration certificate in $\var{stpools}$ of $l'$
and that $hk$ is not in the set of retiring stake pools in $l'$.

\begin{property}[\textbf{Start Staking Pool Retirement}]
  \begin{multline*}
    \forall \var{l}, \var{l'} \in \ledgerState, \var{cepoch} \in \Epoch:
    \applyFun{validLedgerstate}{l},
    l = (\wcard, (\wcard,p)), l' = (\wcard, (\wcard,p')), pEnv\in\PEnv \\
    \implies \forall \var{c} \in \DCertRetirePool, pEnv\vdash\var{p}
    \trans{POOL}{c} \var{p'} \\ \implies e = \applyFun{retire}{c} \wedge
    \var{cepoch} < e < \var{cepoch} + \emax \wedge \applyFun{cwitness}{c} =
    \var{hk}\\ \implies (\applyFun{getRetiring}{p'})[\var{hk}] = e \wedge
    \var{hk} \in
    \fun{dom}(\applyFun{getStPools}{p})\wedge\fun{dom}(\applyFun{getStPools}{p'}
    )
  \end{multline*}
  \label{prop:ledger-properties-10}
\end{property}

Property~\ref{prop:ledger-properties-10} states that for $l, l'$ as above but
with a delegation transition of type $\DCertRetirePool$, the key $hk$ is
associated with the author of the pool registration certificate in
$\var{stpools}$ of $l'$ and that $hk$ is in the map of retiring staking pools of
$l'$ with retirement epoch $e$, as well as that $hk$ is in the map of stake
pools in $l$ and $l'$.

\begin{property}[\textbf{Stake Pool Reaping}]
  \begin{multline*}
    \forall \var{l}, \var{l'} \in \ledgerState, \var{e} \in \Epoch:
    \applyFun{validLedgerstate}{l},\\
    l = (\wcard, (d, p)), l' = (\wcard, (d', p')), pp\in\PParams, acnt, acnt'\in\Acnt \\
    \implies pp\vdash\var{(acnt, d, p} \trans{poolreap}{e} \var{(acnt, d', p')}
    \implies \forall \var{retire}\in{(\fun{getRetiring}~p)}^{-1}[e], retire \neq
    \emptyset \\ \wedge \var{retire} \subseteq
    \fun{dom}(\applyFun{getStPool}{p}) \wedge
    \var{retire} \cap\fun{dom}(\applyFun{getStPool}{p'})=\emptyset \\
    \wedge\var{retire} \cap \fun{dom}(\applyFun{getRetiring}{p'}) = \emptyset
  \end{multline*}
  \label{prop:ledger-properties-11}
\end{property}

Property~\ref{prop:ledger-properties-11} states that for $l, l'$ as above but
with a delegation transition of type POOLREAP, there exist registered stake
pools in $l$ which are associated to stake pool registration certificates and
which are to be retired at the current epoch $\var{e}$. In $l'$ all those stake
pools are removed from the maps $stpools$ and $retiring$.

\subsection{Properties of Numerical Calculations}
\label{sec:prop-numer-calc}

The numerical calculations for refunds and rewards in
(see Section~\ref{sec:epoch}) are also required to have certain properties. In
particular we need to make sure that the functions that use non-integral
arithmetic have properties which guarantee consistency of the system. Here, we
state those properties and formulate them in a way that makes them usable in
properties-based testing for validation in the executable spec.

\begin{property}[\textbf{Minimal Refund}]
  \label{prop:minimal-refund}

  The function $\fun{refund}$ takes a value, a minimal percentage, a decay
  parameter and a duration. It must guarantee that the refunded amount is within
  the minimal refund (off-by-one for rounding / floor) and the original value.

  \begin{multline*}
    \forall d_{val} \in \mathbb{N}, d_{min} \in [0,1], \lambda \in (0, \infty),
    \delta \in \mathbb{N} \\
    \implies \max(0,d_{val}\cdot d_{min} - 1) \leq \floor*{d_{val}\cdot(d_{min} +
      (1-d_{min})\cdot e^{-\lambda\cdot\delta})} \leq d_{val}
  \end{multline*}
\end{property}

\begin{property}[\textbf{Maximal Pool Reward}]
  \label{prop:maximal-pool-reward}

  The maximal pool reward is the expected maximal reward paid to a stake
  pool. The sum of all these rewards cannot exceed the total available reward,
  let $Pool$ be the set of active stake pools:

  \begin{equation*}
    \forall R \in Coin:\sum_{p \in Pools} \floor*{\frac{R}{1+p_{a_{0}}}\cdot
      \left(
        p_{\sigma'}+p_{p'}\cdotp_{a_{0}}\cdot\frac{p_{\sigma'}-p_{p'}\cdot\frac{p_{z_{0}}-p_{\sigma'}}{p_{z_{0}}}}{p_{z_{0}}}
      \right)}\leq R
  \end{equation*}
\end{property}

\begin{property}[\textbf{Actual Reward}]
  \label{prop:actual-reward}

  The actual reward for a stake pool in an epoch is calculated by the function
  $\fun{poolReward}$. The actual reward per stake pool is non-negative and
  bounded by the maximal reward for the stake pool, with $\overline{p}$ being
  the relation $\frac{n}{\max(1, \overline{N})}$ of the number of produced
  blocks $n$ of one pool to the total number $\overline{N}$ of produced blocks
  in an epoch and $maxP$ being the maximal reward for the stake pool. This gives
  us:

  \begin{equation*}
    \forall \gamma \in [0,1] \implies 0\leq \floor*{\overline{p}\cdot maxP} \leq maxP
  \end{equation*}
\end{property}

The two functions $\fun{r_{operator}}$ and $\fun{r_{member}}$ are closely related as
they both split the reward between the pool leader and the members.

\begin{property}[\textbf{Reward Splitting}]
  \label{prop:reward-splitting}

  The reward splitting is done via $\fun{r_{operator}}$ and $\fun{r_{member}}$, i.e.,
  a split between the pool leader and the pool members using the pool cost $c$
  and the pool margin $m$. Therefore the property relates the total reward
  $\hat{f}$ to the split rewards in the following way:

  \begin{multline*}
    \forall m\in [0,1], c\in Coin \implies c + \floor*{(\hat{f} - c)\cdot (m +
      (1 - m)) \cdot \frac{s}{\sigma}} + \sum_{j}\floor*{(\hat{f} -
      c)\cdot(1-m)\cdot\frac{t_{j}}{\sigma}} \leq \hat{f}
  \end{multline*}

\end{property}

\clearpage

%%% Local Variables:
%%% mode: latex
%%% TeX-master: "ledger-spec"
%%% End:
